\subsection{Topics to be discussed}

Our multi-disciplinary discussion will bring together novel ways at
observing the world’s upper water-column bio-geochemical and dynamical
processes. The symposium gathers experts in very different branches of
the science including physical \& biological oceanography, satellite
and in situ sensing, process dynamics \ldots, to ensure we have this
broad and multi-disciplinary bent. The point of the meeting is not to
arrive at a specific conclusion, but to cover a broad range of science
and to see how computational technology especially robotics, AI and
Machine Learning, can and should be adopted. At its core ideation
requires exploration of a range of ideas and not be restrictive, but
to allow ideas to flow freely. 

A preliminary agenda can be found in the annex in Section \ref{sec:annex}.

\subsection{Leading questions}

The overall leading question for all participants will revolve around
how we can \textbf{advance our understanding of the ocean upper
  bio-geochemical and dynamics processes} by using the latest advances
in technology. The implications of this question are far-reaching since
ocean health as well as the well being of society depend on our capacity
to observe, monitor and forecast the bio-geochemical and dynamics
processes in the upper ocean. Layered on top of this will be the
specific set of questions, every speaker will be asked to address:

\begin{enumerate}[noitemsep,topsep=0pt,parsep=0pt,partopsep=0pt]

\item How do/does your idea(s) aid in a smarter way to ocean
  observation? How would this benefit from smarter ocean observation? By
  'smart' we imply doing more with less, and leveraging advances in
  computational sciences including the fields of robotics, AI,
  statistics and engineering.

\item In what way do/does this bridge technology and science for upper
  water column observation? What extensions would strengthen such links
  here to enable sustaining the approach(es) and to scale it up for the
  benefit of the scientific/engineering technology?

\item How might it be adopted by others in the community and change the
  way science is done ?

\item What is the innovation potential? And in turn, what innovations
  are necessary to advance the science/technology?

\item Specifically for the way this symposium is being conducted with a
  pairing of a speaker and critique, what did you learn from the
  process? What were the principal challenges you encountered?

\end{enumerate}

\subsection{Facilitation of the discussion}

The organizers have substantial experience in organizing and
facilitating scientific meetings. We have a deep knowledge of the
topics that will be presented and will interact actively with each
speaker prior to, during and after the event especially given the
non-traditional format for this event. We therefore believe we are in
the best position to facilitate and guide the discussion as and when
necessary and bring it to a meaningful conclusion over the course of
the four days. Prior to the event, we will be engaging all the
invitees well in advance, starting with a virtual Q\&A session on
March 25\textsuperscript{th}. In addition, during the symposium, each
(half day) session will be chaired by a participant, where one of the
organizers and an assigned ESR will be taking notes. In addition, we
will provide all participants with access to simple online tools
such as Google Docs and \texttt{Slido} (\url{https://www.sli.do}) for
note taking and archiving discussions. 

 % In the event that an external facilitator
% would be deemed necessary, and given the length of the symposium (4
% days, 9AM-6PM), it would be necessary to count on at least two external
% facilitators, otherwise it would be too
% exhausting for just one person and this person would under-perform.\\

To augment such efforts, we propose to engage a \emph{visual
  facilitator} (Marsha Dunn \url{https://www.marshadunn.com/}) during
the technical presentations. Such facilitation is efficient in helping
to capture complex ideas visually as a means to start discussion
around certain topics, reflect in real time where the discussions
should lead us and make connections on our reflections that the
participants didn't anticipate. Dunn was an active participant with
one of the organizers at the initiation of the \textbf{OceanX} mission
at the Dalio Foundation in Connecticut in 2018 and was very effective
in post-presentation Q\&A's as well as during breaks for
networking. The visuals produced by Dunn will be made available to all
in attendance and be used in reports and publications with no
additional charge to those attending.
 
\subsection{Note taking}

All three organizers will take turns note-taking and be supplemented by
an ESR as noted above. In addition, all participants will be asked to
use \texttt{Slido} for recording their questions and organize the
discussions. Slido allows everyone to post questions and vote others'
questions in turn, so that we can see in real time which questions
attract more attention and can/should be addressed first.
 
 
\subsection{Synthesizing the attendees’ input}

Notes taken by the organizers and \texttt{Slido} will provide the
reference material for the final report(s) and provide the principal
output immediately post-meeting. The synthesis will be made available to
all participants in final form with a request for their input. In
addition post-meeting we will ask each participant to address questions
directed to them to provide clarity. Visuals produced by Dunn will also
serve as a very efficient summary of the symposium.

 
\subsection{Expected Outputs}

A final report of the event will be submitted to all sponsors within a
month of the symposium.

After the meeting, all authors, led by the organizers will engage in a
\textbf{'Perspective'} piece to be published in a high-profile journal
(one of \emph{Science} or \emph{Nature}. We have also an in-principle
agreement to submit a separate manuscript to \emph{Oceanography}
journal. These academic outputs will likely take about 6 months after
the conclusion of the event.

The end goal and expectation of the organizers is that this meeting
will be a catalyst not only for collaboration amongst peers, but
critically for the invitees have a change in focus in how advances in
computational science have and can impact their research goals in
ocean observation, and for a broader mindset in adopting these modern
methods by engaging the technology community.