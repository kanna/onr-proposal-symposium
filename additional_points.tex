\subsection{Topics Discussed}

Our multi-disciplinary discussion will bring together novel ways at
observing the world’s upper-ocean bio-geochemical and dynamics
processes. The symposium gathers experts in very different branches of
oceanography (physical \& biology, satellite and in situ, robotics, etc)
to ensure we have this broad and multi-disciplinary bent.

A preliminary agenda can be found in Annex I.

\subsection{Leading questions}

The overall leading question for all participants will revolve around
how we can \textbf{advance our understanding of the ocean upper
  bio-geochemical and dynamics processes} by using the latest advances
in technology. The implications of this question are far-reaching since
ocean health as well as the well being of society depend on our capacity
to observe, monitor and forecast the bio-geochemical and dynamics
processes in the upper ocean. Layered on top of this will be the
specific set of questions, every speaker will be asked to address:

\begin{enumerate}

\item How do/does your idea(s) aid in a smarter way to ocean
  observation? How would this benefit from smarter ocean observation? By
  'smart' we imply doing more with less, and leveraging advances in
  computational sciences including the fields of robotics, AI,
  statistics and engineering.

\item In what way do/does this bridge technology and science for upper
  water column observation? What extensions would strengthen such links
  here to enable sustaining the approach(es) and to scale it up for the
  benefit of the scientific/engineering technology?

\item How might it be adopted by others in the community and change the
  way science is done ?

\item What is the innovation potential? And in turn, what innovations
  are necessary to advance the science/technology?

\item Specifically for the way this symposium is being conducted with a
  pairing of a speaker and critique, what did you learn from the
  process? What was the principal challenge in communicating?

\end{enumerate}

\subsection{Facilitation of the discussion}

The organizers have substantial experience in organizing and
facilitating scientific meetings; this is not the first such event,
albeit while doing it together. We have a deep knowledge of the topics
that will be presented and can interact actively with the speakers on
all these. We therefore believe we are in the best position to
facilitate the discussion and bring it to a meaningful conclusion over
the course of the three days. Each session (half day) will be chaired by
a participant, where one of the organizers and an ESR will be taking
notes. In addition, we will provide all participants with access to
simple online tools (e.g. Google Docs) and \texttt{Slido}
(\url{https://www.sli.do}). 

 % In the event that an external facilitator
% would be deemed necessary, and given the length of the symposium (4
% days, 9AM-6PM), it would be necessary to count on at least two external
% facilitators, otherwise it would be too
% exhausting for just one person and this person would under-perform.\\

To augment such efforts, we propose to engage a visual facilitator
(Marsha Dunn \url{https://www.marshadunn.com/}) during the technical
presentations. Such facilitation is efficient in helping to capture
complex ideas visually as a means to start discussion around certain
topics, reflect in real time where the discussions should lead us and
make connections on our reflections that the participants didn't
anticipate. Marsha was an active participant with one of the organizers
at the initiation of the \textbf{OceanX} mission at the Dalio Foundation
in Connecticut in 2018.
 
\subsection{Note taking}

All three organizers will take turns note-taking and be supplemented by
an ESR as noted above. In addition, all participants will be asked to
use \texttt{Slido} for recording their questions and organize the
discussions. Slido allows everyone to post questions and vote others'
questions in turn, so that we can see in real time which questions
attract more attention and can/should be addressed first.
 
 
\subsection{Synthesizing the attendees’ input}

Notes taken by the organizers and \texttt{Slido} will provide the
reference material for the final report(s) and provide the principal
output immediately post-meeting. The synthesis will be made available to
all participants in final form with a request for their input. In
addition post-meeting we will ask each participant to address questions
directed to them to provide clarity. Visuals produced by Dunn will also
serve as a very efficient summary of the symposium.

 
\subsection{Deliverable report}

After the meeting, all authors, led by the organizers will engage in a
"perspective" paper to be published in a high-profile journal (one of
\emph{Science, Nature} or \emph{Oceanography}).

